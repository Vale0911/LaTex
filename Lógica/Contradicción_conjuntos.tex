\documentclass{article}
\usepackage{tikz}
\usetikzlibrary{patterns}
\usepackage{pgfplots}
\pgfplotsset{compat=1.16}
\usepackage[spanish]{babel}
\usepackage{amsmath, amssymb, graphics, setspace}

% Set page size and margins
% Replace `letterpaper' with `a4paper' for UK/EU standard size
\usepackage[letterpaper,top=2cm,bottom=3cm,left=2cm,right=3cm,marginparwidth=1.75cm]{geometry}
\usepackage{multicol}
\usepackage{amsmath}
\usepackage{amsthm}
\usepackage{fancyhdr}
\usepackage{graphicx}
\usepackage{parskip}
\usepackage{blindtext}
\usepackage{dirtytalk}
\usepackage{tikz}
\usetikzlibrary{arrows}
\usepackage[colorlinks=true, allcolors=blue]{hyperref}

\theoremstyle{definition}
\newtheorem{definition}{Definición}[section]


\singlespacing
\renewcommand{\headrulewidth}{0pt}
\fancyhead[C]{
    \includegraphics[width=6cm]{Logo_URosario.png}
}
\title{Universidad del Rosario\\
\textbf{Lógica, Teoría de Números y Conjuntos}\\
\textbf{\Large Tarea 2}\\
\textbf{\large Profesor: César Augusto Rodríguez Duque}\\
\author{Valentina Mesa Fajardo - 1021397618 - valentina.mesa@urosario.edu.co}}
\date{\small\today}

\begin{document}
\maketitle
\thispagestyle{fancy}
\singlespacing
\large

\begin{enumerate}
  \item (Ejercicio 17) Demuestre por contradicción que si $n\in\mathbb{Z}$ y $11|n^{2}$, entonces $11|n$. 
  
  {\centering \textbf{Desarrollo:}\\}
  \textit{Demostrar por contradicción}\\
  Supongamos lo contrario, si $11|n^{2}$ entonces $11\nmid n$. Si $11\nmid n$, n no es múltiplo de 11.\\
  \textit Como $11\nmid n$ su residuo debe ser distinto de 0, tal que: $n=11K +r$ para algún $K y r \in \mathbb{Z}$. Donde K es el cociente y r el residuo.\\
  
   \textit Ya que nos interesa $n^{2}$, elevamos a ambos lados de la ecuación al cuadrado.\\
    \begin{alignat*}{2}
            n^{2}&=(11k+r)^{2}     &\quad &\text{Expandimos:}\\
                 &=(11k+r)(11k+r)  &\quad &\text{binomio al cuadrado}\\
                 &=121k^{2}+11kr+11kr+r^{2} &\quad &\text{resolvemos términos comunes.}\\
                 &=121k^{2}+22kr+r^{2}       &\quad &\text{}\\
  \end{alignat*}
  \textit Ahora observemos que sucede al dividir estos factores ($121k^{2}+22kr+r^{2}$) por 11, es decir, hallar los residuos de $n^{2}$ mod 11:\\
      \begin{alignat*}{2}
            121^{2} \ mod \ 11 &=    &\quad &\text{(es divisible por 11) su residuo es 0}\\
            22kr\ mod \ 11 &=       &\quad &\text{(es divisible por 11) su residuo es 0}\\
  \end{alignat*}
  \textit Lo que nos deja con una expresión simplificada.\\
        \begin{alignat*}{2}
            n^{2} &= r^{2} \ mod \ 11   &\quad &\text{}\\
  \end{alignat*}
    \textit Es decir, el residuo de $n^{2}$ es el mismo que el de $r^{2}$ (al dividirlos por 11). Evaluemos los valores arbitrarios de r:\\
    \begin{alignat*}{2}
            r &= 1 &= r^{2} &= (1)^{2} &= 1\ mod \ 11 &=  0,09 &\quad &\text{}\\
            r &= 5 &= r^{2} &= (5)^{2} &= 25\ mod \ 11 &=  2,27 &\quad &\text{}\\
            r &= 9 &= r^{2} &= (9)^{2} &= 81\ mod \ 11 &=  2,27 &\quad &\text{}\\
\end{alignat*}
    \textit Ningún residuo es $0$, si $r\neq 0 $ $(\rightarrow\leftarrow)$. Ningún valor de $r^{2}$ $mod\ 11=0$, nos lleva a una contradicción, habíamos supuesto que $11\nmid n$. Por lo tanto si $11|n^{2}$, entonces $11|n$\\
 \item (Ejercicio 20) Demuestre por contradicción que la suma de un número irracional y un número racional es un número irracional. 
  
  {\centering \textbf{Desarrollo:}\\}
  \textit{Demostrar por contradicción}\\
  Supongamos lo contrario, la suma de un número racional con un número irracional, es un número racional.\\
  \textit Tenemos un racional $a$ y un irracional $b$, $a+b=p$ (un número racional) para algún $p\in\mathbb{Z}$. Entonces $b=a-p$ como a y p son racionales, b tiene que ser racional $(\rightarrow\leftarrow)$.  \\

 \item (Ejercicio 25) Usando como referencia los números enteros; determinar aquellos números que satisfacen la desigualdad $n^{2}<49$, expresar este conjunto tanto por extensión como por comprensión.
 
 {\centering \textbf{Desarrollo:}\\}
  Aplicamos raíz cuadrada a ambos lados:
  $$\sqrt{n^{2}}<\sqrt{49}$$
  Como $\sqrt{x^{2}}=|x|$:
  $$|n|<7$$
  n puede ser $7 o -7$:
  $$-7<n<7$$
  Esto divide a la inecuación en el siguiente intervalo:
  $$-7<n<7$$
  Al expresarlo en conjuntos, tenemos: 
  $$\textit{Por extensión}\\$$
  $$S=\{-6,-5,-4,-3,-2,-1,0,1,2,3,4,5,6\}$$
  $$\textit{Por comprensión}$$
  $$S=\{n\in\mathbb{Z}: -7 < n < 7\}$$

  \item (Ejercicio 27) Si $A, B$ y $C$ son conjuntos, demuestre o refute las siguientes igualdades entre conjuntos:
  \begin{multicols}{2}
    \begin{itemize}
        \item $A\cup\varnothing=\varnothing$
        \item $A\cap\varnothing=\varnothing$
        \item $(A\cup B)\smallsetminus B = A$
        \item $(A\smallsetminus B)\cup B = A$
        \item $(A\smallsetminus C)\cup(B\smallsetminus C)=A\cap B$
        \item $A\smallsetminus(B\cup C)=(A\smallsetminus B)\cup(A\smallsetminus C)$
    \end{itemize}
  \end{multicols}
  {\centering \textbf{Desarrollo:}\\}
  \begin{itemize}
      \item $A\cup\varnothing=\varnothing$\\
      Se tiene,
      $$A\cup\varnothing=\{x: x\in A \lor x\in\varnothing\}=\{x:x\in A\}$$
      Es decir,
      $$A\cup\varnothing=A$$
      Por lo tanto $A\cup\varnothing=\varnothing$ es verdad si y solo si $A=\varnothing$, de lo contrario es \textbf{falso}.
       \item $A\cap\varnothing=\varnothing$\\
      Se tiene,
      $$A\cap\varnothing=\{x: x\in A \lor x\in\varnothing\}$$
      Como no hay elementos en $\varnothing$, no hay un $x$ que pueda satisfacer esta condición "$x\in A \lor x\in\varnothing$". Por lo tanto es \textbf{verdadero}.
      
      \item $(A\cup B)\smallsetminus B = A$\\
      Se tiene,
      $$(A \cup B)\smallsetminus B=\{x: x\in (A\cup B) \wedge x\notin b\}$$
      debido a la unión de conjuntos:
      $$(A\cup B)\smallsetminus B =\{x: (x\in A \lor x\in B) \wedge x\notin B\}$$
      Si $x\in A$, entonces se cumple que $x\in (A\cup B)$ y $x\in B$, ya que $x\notin B$.
      \textit Si $x\in B$, no se cumple que $x\in (A\cup B)$ ya que $x\notin B$. Por lo tanto, $(A\cup B)\smallsetminus B=\{x: x\in A\}$ (el conjunto A) entonces la igualdad es \textbf{verdadera}. 

      \item $(A\smallsetminus B)\cup B = A$\\
      Se tiene, 
      $$(A \smallsetminus B) \cup B=\{x: x\in (A\smallsetminus B)\lor x\in B\} $$
      Por lo tanto,
      $$(A \smallsetminus B) \cup B=\{x: (x\in A \wedge x\notin B) \lor x\in B\}$$
      Para cualquier elemento $x$ del conjunto $(A \smallsetminus B)\cup B$ tenemos dos casos.
      $$\textit{Caso 1.}$$
      Si $x\in A \wedge x\notin B$, claramente $x\in A$ únicamente.
      $$\textit{Caso 2.}$$
      Si $x\in B$, no importa si $x\in A$ o no, la unión $(A \smallsetminus B)\cup B$ incluye todos los elementos de $B$. Sin embargo como $x\in B \subseteq (A \cup B$), asegurando que $x$ también esta en $A$ si está en A, la unión incluye todos los elementos de $A$ y $B$.Por lo tanto es \textbf{Verdadero}, cada elemento en $(A \smallsetminus B)\cup B$ está en $A$.\\

      \item $(A\smallsetminus C)\cup(B\smallsetminus C)=A\cap B$\\
      \textit{Contraejemplo}
      Consideramos ejemplos específicos y verificamos si la igualdad se cumple.
      $$A \smallsetminus C = \{x: x\in A \lor x\notin C\}=\{1,3,4\}$$
      $$B \smallsetminus C = \{x: x\in B \lor x\notin C\}=\{3,4,6\}$$
      Calculamos la unión.
      $$(A\smallsetminus C) \cup (B\smallsetminus C)=\{1,3,4,6\}$$
      Calculamos la intersección.
      $$A \cap B = \{3,4\}$$
      Como se observa, $(A \smallsetminus C)\cup (B \smallsetminus C)=\{1,3,4,6\}$ no es igual a $A \cap B = \{3,4\}$, por lo tanto es \textbf{falso}.

      \item $A\smallsetminus(B\cup C)=(A\smallsetminus B)\cup(A\smallsetminus C)$.\\
      Lado izquierdo.
      $$A \smallsetminus (B \cup C)=\{x:(x\in A \wedge x\notin (B \cup C)\}$$
      Lado derecho.
      $$(A \smallsetminus B)\cup(A \smallsetminus C)=\{x: (x\in A \wedge x\notin B)\lor\ (x\in A \wedge x\notin B)\}$$
      Consideremos ejemplos específicos para ver diferencias en ambos lados de la igualdad. 
      $$A=\{1,2,3\} \\\\\\\ B=\{2\} \\\\\\\ C=\{3\}$$
      Lado izquierdo.
      $$A \smallsetminus(B \cup C)$$
      $$(B \cup C) =\{2,3\} \Longrightarrow A\smallsetminus (B\cup C)=\{1\}$$
      Lado derecho. 
      $$(A \smallsetminus B)\cup (A \smallsetminus C)$$
      $$(A \smallsetminus C)=\{1,2\}$$
      $$(A \smallsetminus B)=\{1,3\}$$
      $$(A \smallsetminus B)\cup (A \smallsetminus C =\{1,2,3\})$$
      Por lo tanto, es \textbf{falso}, ya que, $A \smallsetminus (B\cup C)\neq (A \smallsetminus B)\cup (A \smallsetminus C)$. 
      
 
  \end{itemize}

\end{enumerate}

\end{document}
