\documentclass{article}
\usepackage{graphicx} % Required for inserting images
\usepackage{listings} % Paquete para código
\usepackage{xcolor}   % Paquete para colores en el código

\title{Prime Jack - hitos y prompts}
\author{Valentina Mesa}
\date{September 2024}

% Configuración de estilo para el código
\lstset{
    basicstyle=\ttfamily\footnotesize,
    keywordstyle=\color{blue},
    commentstyle=\color{gray},
    stringstyle=\color{orange},
    numbers=left,
    numberstyle=\tiny\color{gray},
    stepnumber=1,
    numbersep=10pt,
    frame=single,
    tabsize=4,
    showstringspaces=false
}
\usepackage[colorlinks=true, linkcolor=blue, urlcolor=blue, citecolor=blue]{hyperref}

\begin{document}

\maketitle

\section{Hito. Estructura inicial del código en Python y c++}


\textbf{Prompt Usado:}

\begin{quote}
“Eres un desarrollador y te contratan para realizar un juego en Python y C++. El juego consiste en lo siguiente:

En lugar de cartas, se usan 2 dados de 6 lados. El objetivo es llegar a 23 o al número más cercano, priorizando números primos. Un número primo gana frente a uno mayor. Entre primos, gana el mayor.

Si un jugador saca dobles, puede dividir el lanzamiento y usar 3 dados adicionales para cada nueva tirada.

Si sale un 12, el jugador puede lanzar solo 1 dado más, con un máximo de 18.”
\end{quote}

\textbf{Resultado:}
\begin{quote}
ChatGPT generó un código con términos que aún no comprendo bien, lo que lo hace algo complejo. Utiliza decoradores para penalizar automáticamente a los jugadores que exceden el valor permitido, además de listas por comprensión y funciones lambda para simplificar operaciones. La implementación con clases también mejora la estructura del código.

    
\end{quote}

\begin{lstlisting}[language=Python, caption=Código en Python]
import random

def penalizar_si_excede(func):
    return lambda *args, **kwargs: 
    max(func(*args, **kwargs) - random.randint(1, 5), 0) 
    if func(*args, **kwargs) > 23 else func(*args, **kwargs)

class Jugador:
    def __init__(self, nombre):
        self.nombre = nombre
        self.puntaje = 0

    @penalizar_si_excede
    def jugar_turno(self):
        dados = [random.randint(1, 6) for _ in range(2)]
        self.puntaje = sum(dados)
        if dados[0] == dados[1]:
            self.puntaje = sum([random.randint(1, 6) for _ in range(3)])
        if self.puntaje == 12:
            self.puntaje += random.randint(1, 6)
        return self.puntaje

def es_primo(num):
    return num > 1 and all(num % i for i in range(2, int(num**0.5) + 1))

def determinar_ganador(j1, j2):
    if es_primo(j1.puntaje) != es_primo(j2.puntaje):
        return f"{j1.nombre} gana con un número primo." 
        if es_primo(j1.puntaje) else f"{j2.nombre} gana con un número primo."
    return f"{j1.nombre} gana." if j1.puntaje > j2.puntaje else f"{j2.nombre} gana."

jugador1, jugador2 = Jugador("Jugador 1"), Jugador("Jugador 2")
print(determinar_ganador(jugador1, jugador2))

\end{lstlisting}


\begin{lstlisting}[language=C++, caption=Código en C++]
#include <iostream>
#include <vector>
#include <cstdlib>  // Para rand() y srand()
#include <ctime>    // Para time()
#include <cmath>    // Para sqrt()

using namespace std;

// Función para penalizar si el puntaje excede 23
int penalizar_si_excede(int puntaje) {
    if (puntaje > 23) {
        int penalizacion = rand() % 5 + 1; // Genera un número aleatorio entre 1 y 5
        return max(puntaje - penalizacion, 0);
    }
    return puntaje;
}

// Clase Jugador
class Jugador {
public:
    string nombre;
    int puntaje;

    Jugador(string nombre) : nombre(nombre), puntaje(0) {}

    int jugar_turno() {
        vector<int> dados = {rand() % 6 + 1, rand() % 6 + 1}; // Dos dados de 1 a 6
        puntaje = dados[0] + dados[1];

        // Si los dados son iguales (dobles), tirar 3 dados adicionales
        if (dados[0] == dados[1]) {
            puntaje = 0;
            for (int i = 0; i < 3; ++i) {
                puntaje += rand() % 6 + 1;
            }
        }

        // Si el puntaje es 12, tirar 1 dado más
        if (puntaje == 12) {
            puntaje += rand() % 6 + 1;
        }

        // Aplicar penalización si excede 23
        puntaje = penalizar_si_excede(puntaje);

        return puntaje;
    }
};

// Función para determinar si un número es primo
bool es_primo(int num) {
    if (num <= 1) return false;
    for (int i = 2; i <= sqrt(num); ++i) {
        if (num % i == 0) return false;
    }
    return true;
}

// Función para determinar el ganador
string determinar_ganador(Jugador& j1, Jugador& j2) {
    if (es_primo(j1.puntaje) != es_primo(j2.puntaje)) {
        if (es_primo(j1.puntaje)) {
            return j1.nombre + " gana con un número primo.";
        } else {
            return j2.nombre + " gana con un número primo.";
        }
    }

    return (j1.puntaje > j2.puntaje) ? j1.nombre + " gana." : j2.nombre + " gana.";
}

int main() {
    // Semilla para generar números aleatorios
    srand(time(0));

    // Crear los jugadores
    Jugador jugador1("Jugador 1");
    Jugador jugador2("Jugador 2");

    // Cada jugador juega su turno
    jugador1.jugar_turno();
    jugador2.jugar_turno();

    // Determinar y mostrar el ganador
    cout << determinar_ganador(jugador1, jugador2) << endl;

    return 0;
}

}
\end{lstlisting}
\section{Hito. Simplificación del código en Python y c++}


\textbf{Prompt Usado:}

\begin{quote}
“Solo puedes usar primitivas básicas. Condicionales, Ciclos,etc.”
\end{quote}

\textbf{Resultado:}
\begin{quote}
Con este Promt, chat gpt mejoró el codigo usando funciones simples como \verb|es_primo| para verificar números primos, \verb|lanzar_dados| para lanzar dados, y \verb|jugar_ronda| para gestionar el turno de juego. Este enfoque es más directo y fácil de entender porque cada función realiza una tarea específica sin la complejidad de clases o decoradores. Sin embargo, el juego no corría como yo esperaba. 

    
\end{quote}

\begin{lstlisting}[language=Python, caption=Código en Python]
import random

# Función para verificar si un número es primo
def es_primo(n):
    if n < 2:
        return False
    for i in range(2, int(n**0.5) + 1):
        if n % i == 0:
            return False
    return True

# Función para lanzar dados
def lanzar_dados(cantidad=2):
    return [random.randint(1, 6) for _ in range(cantidad)]

# Función para jugar una ronda
def jugar_ronda():
    total = 0
    while total < 23:
        dados = lanzar_dados()
        print(f"Lanzaste: {dados}")
        
        # Si el jugador saca dobles
        if dados[0] == dados[1]:
            print(f"¡Sacaste dobles de {dados[0]}!")
            # Permite dividir el lanzamiento
            for _ in range(2):  # Dos lanzamientos adicionales
                nuevos_dados = lanzar_dados(3)  # Se lanzan 3 dados
                print(f"Nuevos dados tras dividir: {nuevos_dados}")
                total += sum(nuevos_dados)
                if total >= 23:
                    break
            continue
        
        # Si el jugador saca 12
        if sum(dados) == 12:
            print("¡Sacaste un 12! Puedes lanzar un solo dado.")
            dado_extra = lanzar_dados(1)[0]
            print(f"Lanzaste el dado extra: {dado_extra}")
            total += dado_extra
            if total >= 23:
                break
            continue

        total += sum(dados)

        # Opción de detener el juego si se llega a 23 o más
        if total >= 23:
            break
        
        print(f"Total acumulado: {total}")
    
    return total

# Función para determinar el ganador
def determinar_ganador(jugador1, jugador2):
    # Revisar si alguno tiene un número primo
    if es_primo(jugador1) and es_primo(jugador2):
        return 1 if jugador1 > jugador2 else 2
    if es_primo(jugador1):
        return 1
    if es_primo(jugador2):
        return 2
    
    # Si ninguno tiene primos, gana el que esté más cerca a 23
    return 1 if jugador1 <= 23 and (jugador1 > jugador2 or jugador2 > 23) else 2

# Juego principal
def juego_prime_jack():
    print("Turno del Jugador 1")
    jugador1 = jugar_ronda()
    print(f"Puntuación final del Jugador 1: {jugador1}\n")

    print("Turno del Jugador 2")
    jugador2 = jugar_ronda()
    print(f"Puntuación final del Jugador 2: {jugador2}\n")

    ganador = determinar_ganador(jugador1, jugador2)
    print(f"El ganador es el Jugador {ganador}")

# Ejecutar el juego
juego_prime_jack()


\end{lstlisting}

\begin{lstlisting}[language=C++, caption=Código en C++]
#include <iostream>
#include <vector>
#include <cstdlib>  // Para rand() y srand()
#include <ctime>    // Para time()
#include <cmath>    // Para sqrt()

using namespace std;

// Función para penalizar si el puntaje excede 23
int penalizar_si_excede(int puntaje) {
    if (puntaje > 23) {
        int penalizacion = rand() % 5 + 1; // Genera un número aleatorio entre 1 y 5
        return max(puntaje - penalizacion, 0);
    }
    return puntaje;
}

// Clase Jugador
class Jugador {
public:
    string nombre;
    int puntaje;

    Jugador(string nombre) : nombre(nombre), puntaje(0) {}

    int jugar_turno() {
        vector<int> dados = {rand() % 6 + 1, rand() % 6 + 1}; // Dos dados de 1 a 6
        puntaje = dados[0] + dados[1];

        // Si los dados son iguales (dobles), tirar 3 dados adicionales
        if (dados[0] == dados[1]) {
            puntaje = 0;
            for (int i = 0; i < 3; ++i) {
                puntaje += rand() % 6 + 1;
            }
        }

        // Si el puntaje es 12, tirar 1 dado más
        if (puntaje == 12) {
            puntaje += rand() % 6 + 1;
        }

        // Aplicar penalización si excede 23
        puntaje = penalizar_si_excede(puntaje);

        return puntaje;
    }
};

// Función para determinar si un número es primo
bool es_primo(int num) {
    if (num <= 1) return false;
    for (int i = 2; i <= sqrt(num); ++i) {
        if (num % i == 0) return false;
    }
    return true;
}

// Función para determinar el ganador
string determinar_ganador(Jugador& j1, Jugador& j2) {
    if (es_primo(j1.puntaje) != es_primo(j2.puntaje)) {
        if (es_primo(j1.puntaje)) {
            return j1.nombre + " gana con un número primo.";
        } else {
            return j2.nombre + " gana con un número primo.";
        }
    }

    return (j1.puntaje > j2.puntaje) ? j1.nombre + " gana." : j2.nombre + " gana.";
}

int main() {
    // Semilla para generar números aleatorios
    srand(time(0));

    // Crear los jugadores
    Jugador jugador1("Jugador 1");
    Jugador jugador2("Jugador 2");

    // Cada jugador juega su turno
    jugador1.jugar_turno();
    jugador2.jugar_turno();

    // Determinar y mostrar el ganador
    cout << determinar_ganador(jugador1, jugador2) << endl;

    return 0;
}

}
\end{lstlisting}
\section{Hito. Correcciones del código en Python y c++}


\textbf{Prompt Usado:}

\begin{quote}
Como el juego con el anterior código creaba la partida completa con solo ejecutar el codigo, le pedi a chat gpt lo siguiente: “El jugador debe escoger si desea lanzar otra vez o quedarse con el puntaje. Corrige eso.”
\end{quote}

\textbf{Resultado:}
\begin{quote}
Chat gpt, creó la variable \verb|seguir_jugando| que controla si el jugador desea continuar lanzando dados o detenerse. Después de cada lanzamiento, se pregunta al jugador si quiere seguir o quedarse con el total actual; si elige no continuar o si el total alcanza o supera 23, el turno termina. Estos ajustes permiten a los jugadores decidir si quieren seguir jugando para aumentar su puntuación o detenerse para protegerla.
\end{quote}
\begin{lstlisting}[language=Python, caption=Código en Python]
import random

# Función para verificar si un número es primo
def es_primo(n):
    if n < 2:
        return False
    for i in range(2, int(n**0.5) + 1):
        if n % i == 0:
            return False
    return True

# Función para lanzar dados
def lanzar_dados(cantidad=2):
    return [random.randint(1, 6) for _ in range(cantidad)]

# Función para jugar una ronda
def jugar_ronda():
    total = 0
    seguir_jugando = True

    while seguir_jugando and total < 23:
        dados = lanzar_dados()
        print(f"Lanzaste: {dados}")
        
        # Si el jugador saca dobles
        if dados[0] == dados[1]:
            print(f"¡Sacaste dobles de {dados[0]}!")
            # Permite dividir el lanzamiento
            for _ in range(2):  # Dos lanzamientos adicionales
                nuevos_dados = lanzar_dados(3)  # Se lanzan 3 dados
                print(f"Nuevos dados tras dividir: {nuevos_dados}")
                total += sum(nuevos_dados)
                if total >= 23:
                    break
            continue
        
        # Si el jugador saca 12
        if sum(dados) == 12:
            print("¡Sacaste un 12! Puedes lanzar un solo dado.")
            dado_extra = lanzar_dados(1)[0]
            print(f"Lanzaste el dado extra: {dado_extra}")
            total += dado_extra
            if total >= 23:
                break
            continue

        total += sum(dados)

        print(f"Total acumulado: {total}")
        
        # Preguntar al jugador si desea seguir lanzando o quedarse con su total actual
        if total < 23:
            respuesta = input("¿Quieres lanzar de nuevo? (s/n): ").strip().lower()
            if respuesta != 's':
                seguir_jugando = False

    return total

# Función para determinar el ganador
def determinar_ganador(jugador1, jugador2):
    # Revisar si alguno tiene un número primo
    if es_primo(jugador1) and es_primo(jugador2):
        return 1 if jugador1 > jugador2 else 2
    if es_primo(jugador1):
        return 1
    if es_primo(jugador2):
        return 2
    
    # Si ninguno tiene primos, gana el que esté más cerca a 23
    return 1 if jugador1 <= 23 and (jugador1 > jugador2 or jugador2 > 23) else 2

# Juego principal
def juego_prime_jack():
    print("Turno del Jugador 1")
    jugador1 = jugar_ronda()
    print(f"Puntuación final del Jugador 1: {jugador1}\n")

    print("Turno del Jugador 2")
    jugador2 = jugar_ronda()
    print(f"Puntuación final del Jugador 2: {jugador2}\n")

    ganador = determinar_ganador(jugador1, jugador2)
    print(f"El ganador es el Jugador {ganador}")

# Ejecutar el juego
juego_prime_jack()

\end{lstlisting}
\begin{lstlisting}[language=C++, Caption=Código en C++]
    #include <iostream>
#include <cstdlib>  // Para rand() y srand()
#include <ctime>    // Para time()

using namespace std;

// Función para generar un número aleatorio en un rango
int generar_numero_aleatorio(int min, int max) {
    return min + rand() % (max - min + 1);
}

// Función para verificar si un número es primo
bool es_primo(int num) {
    if (num <= 1) return false;
    for (int i = 2; i * i <= num; ++i) {
        if (num % i == 0) return false;
    }
    return true;
}

// Función para jugar una ronda
int jugar_ronda() {
    int total = 0;
    while (total < 23) {
        int dado1 = generar_numero_aleatorio(1, 6);
        int dado2 = generar_numero_aleatorio(1, 6);
        cout << "Lanzaste: " << dado1 << " y " << dado2 << endl;

        if (dado1 == dado2) {
            cout << "¡Sacaste dobles de " << dado1 << "!" << endl;
            for (int i = 0; i < 3; ++i) {
                int dado_extra = generar_numero_aleatorio(1, 6);
                cout << "Nuevo dado: " << dado_extra << endl;
                total += dado_extra;
            }
        } else if (dado1 + dado2 == 12) {
            cout << "¡Sacaste un 12! Puedes lanzar un solo dado." << endl;
            int dado_extra = generar_numero_aleatorio(1, 6);
            cout << "Dado extra: " << dado_extra << endl;
            total += dado_extra;
        } else {
            total += dado1 + dado2;
        }

        if (total >= 23) break;
        cout << "Total acumulado: " << total << endl;
    }

    return total;
}

// Función para determinar el ganador
int determinar_ganador(int puntaje1, int puntaje2) {
    bool primo1 = es_primo(puntaje1);
    bool primo2 = es_primo(puntaje2);

    if (primo1 && !primo2) return 1;
    if (!primo1 && primo2) return 2;
    if (primo1 && primo2) return (puntaje1 > puntaje2) ? 1 : 2;
    return (puntaje1 > puntaje2) ? 1 : 2;
}

int main() {
    srand(static_cast<unsigned int>(time(0)));

    cout << "Turno del Jugador 1" << endl;
    int puntaje1 = jugar_ronda();
    cout << "Puntuación final del Jugador 1: " << puntaje1 << endl;

    cout << "Turno del Jugador 2" << endl;
    int puntaje2 = jugar_ronda();
    cout << "Puntuación final del Jugador 2: " << puntaje2 << endl;

    int ganador = determinar_ganador(puntaje1, puntaje2);
    cout << "El ganador es el Jugador " << ganador << endl;

    return 0;
}

\end{lstlisting}
\section{Hito. Correcciones en c++}


\textbf{Prompt Usado:}

\begin{quote}
"Elimina algunas bibliotecas y usa solo las funcionalidades básicas necesarias en c++”
\end{quote}

\textbf{Resultado:}
\begin{quote}
Las bibliotecas se redujeron a solo \verb|<iostream>|, \verb|<cstdlib>|, y \verb|<ctime>|, y la generación de números aleatorios se simplifica con la función \verb|generar_numero_aleatorio|. Se eliminaron \verb|vector| y \verb|cmath|, utilizando solo operaciones básicas de C++. El código resultante es más lineal y directo, con menos funciones y más centrado en las operaciones fundamentales del juego.
\end{quote}
\begin{lstlisting}[language=C++, Caption=Código en C++]
#include <iostream>
#include <cstdlib>  // Para rand() y srand()
#include <ctime>    // Para time()

using namespace std;

// Función para generar un número aleatorio en un rango
int generar_numero_aleatorio(int min, int max) {
    return min + rand() % (max - min + 1);
}

// Función para verificar si un número es primo
bool es_primo(int num) {
    if (num <= 1) return false;
    for (int i = 2; i * i <= num; ++i) {
        if (num % i == 0) return false;
    }
    return true;
}

// Función para jugar una ronda
int jugar_ronda() {
    int total = 0;
    while (total < 23) {
        int dado1 = generar_numero_aleatorio(1, 6);
        int dado2 = generar_numero_aleatorio(1, 6);
        cout << "Lanzaste: " << dado1 << " y " << dado2 << endl;

        if (dado1 == dado2) {
            cout << "¡Sacaste dobles de " << dado1 << "!" << endl;
            for (int i = 0; i < 3; ++i) {
                int dado_extra = generar_numero_aleatorio(1, 6);
                cout << "Nuevo dado: " << dado_extra << endl;
                total += dado_extra;
            }
        } else if (dado1 + dado2 == 12) {
            cout << "¡Sacaste un 12! Puedes lanzar un solo dado." << endl;
            int dado_extra = generar_numero_aleatorio(1, 6);
            cout << "Dado extra: " << dado_extra << endl;
            total += dado_extra;
        } else {
            total += dado1 + dado2;
        }

        if (total >= 23) break;
        cout << "Total acumulado: " << total << endl;
    }

    return total;
}

// Función para determinar el ganador
int determinar_ganador(int puntaje1, int puntaje2) {
    bool primo1 = es_primo(puntaje1);
    bool primo2 = es_primo(puntaje2);

    if (primo1 && !primo2) return 1;
    if (!primo1 && primo2) return 2;
    if (primo1 && primo2) return (puntaje1 > puntaje2) ? 1 : 2;
    return (puntaje1 > puntaje2) ? 1 : 2;
}

int main() {
    srand(static_cast<unsigned int>(time(0)));

    cout << "Turno del Jugador 1" << endl;
    int puntaje1 = jugar_ronda();
    cout << "Puntuación final del Jugador 1: " << puntaje1 << endl;

    cout << "Turno del Jugador 2" << endl;
    int puntaje2 = jugar_ronda();
    cout << "Puntuación final del Jugador 2: " << puntaje2 << endl;

    int ganador = determinar_ganador(puntaje1, puntaje2);
    cout << "El ganador es el Jugador " << ganador << endl;

    return 0;
}


\end{lstlisting}
\section*{Fortalezas y debilidades.}
\begin{quote}
    El uso de la IA fue muy útil para generar y simplificar el código, proporcionándome soluciones y explicaciones rápidas. Sin embargo, al principio, el código generado era complejo debido al uso de técnicas avanzadas como decoradores y listas por comprensión, lo cual complicó mi entendimiento.
\end{quote}
\textbf{Pros}
\begin{quote}
    La simplificación del código mediante el uso de funciones básicas y condicionales fue fundamental. Esto facilitó la comprensión y la implementación del juego.

\end{quote}
\textbf{Contras}
\begin{quote}
    El código inicial en Python y C++ contenía elementos avanzados que no comprendía completamente. Esto hizo que la implementación y depuración fueran más difíciles hasta que con la ayuda de la misma IA logré simplificarlo.
\end{quote}
\section{Problemas enfrentados y soluciones}
\begin{quote}
    El código inicial era complejo por usar decoradores y listas por comprensión, lo que simplifiqué eliminando estos elementos y utilizando solo funciones básicas. Además, modifiqué la implementación para permitir a los jugadores decidir si continuar o detenerse mediante la variable \verb|seguir_jugando|, y reduje las bibliotecas en C++ a las esenciales \verb|<iostream>|, \verb|<cstdlib>|, y \verb|<ctime>|, simplificando así la generación de números aleatorios.
\end{quote}
\section{Cómo funciona mi código?}
\begin{quote}
    En Python, el código usa varias funciones clave como: \verb|es_primo(n)| para verificar números primos, \verb|lanzar_dados(cantidad=2)|para lanzar dados y \verb|jugar_ronda()| para gestionar una ronda del juego. La función \verb|determinar_ganador(jugador1, jugador2)| decide el ganador basándose en números primos o el puntaje más cercano a 23. Finalmente, \verb|juego_prime_jack()| coordina el juego entre dos jugadores y determina el ganador.
\end{quote}
\begin{quote}
    Así mismo, En C++, el código usa \verb|generar_numero_aleatorio(min, max)| para generar números aleatorios, \verb|es_primo(num)| para verificar números primos y \verb|jugar_ronda()| para manejar el juego, incluyendo lanzamientos de dados y casos especiales. La función \verb|determinar_ganador|(puntaje1, puntaje2) usa booleanos para comparar si las puntuaciones son primas y decide el ganador. Finalmente, \verb|main()| ejecuta el juego, gestionando turnos y mostrando el resultado.
\end{quote}
Link juego c++ \href{https://github.com/Vale0911/c-2/blob/master/TallerIA_Ciclos/prime_jack23.cpp}{github}.

Link juego python \href{https://github.com/Vale0911/python/blob/master/TallerIA_Ciclos/Prime_jack23.py}{github}.


\end{document}